\documentclass[titlepage, 12pt]{article}
\usepackage[T1]{fontenc}
\usepackage[utf8]{inputenc}
\usepackage[francais]{babel}
\usepackage{amssymb}

% Paragraph formatting
\setlength{\parindent}{0pt}
\usepackage{setspace}
\onehalfspacing
% Margin
\usepackage[margin=1in]{geometry}

% Redefine emph to also bold text
\let\emph\relax
\DeclareTextFontCommand{\emph}{\bfseries\em}

% Some chars
\newcommand\BS{\char`\\}
\newcommand\HASH{\char`\#}
\newcommand\SCMCAR{\BS\HASH}

% Title formatting
\usepackage{titlesec}
\titleformat*{\section}{\large\bfseries}
\renewcommand\thesection{\arabic{section}.}
\titleformat*{\subsection}{\normalsize\bfseries}
\renewcommand\thesubsection{(\alph{subsection})}

% Doc info
\title{IFT2035 A2016 \\ TP2 Rapport}
\author{Alizée Gagnon \and Guillaume Noël-Martel}

\begin{document}

\maketitle
Pour le deuxième travail pratique du cours de Concepts en langages de programmation, il était demandé de produire une calculatrice postfixe à précision infinie en Scheme. Nous devions construire notre programme par dessus une ébauche de code qui gérait les entrées et sorties. Nous nous en sommes sortis en utilisant le principe d'une calculatrice à pile et en tirant partie des particularités de Scheme. 
\section{Fonctionnement général du programme}

L'utilisateur écrit une expression mathématique qui est envoyée à la fonction \emph{traiter} par la fonction \emph{repl}, avec un dictionnaire contenant les variables initialisées. \\
Dans la fonction traiter, on  veut trouver le résultat de l'expression. Ceci est possible en effectuant le calcul représenté par l'expression sur une pile. Après avoir transformé l'expression en une liste d'opérations appelées \emph{tokens}, on veut appliquer chaque token sur la pile, ce qui est simplifié en utilisant la fonction \emph{foldl} sur la liste de tokens. La base envoyée au \emph{foldl} est une variable appelée \emph{state}, qui contient à la fois la pile et le dictionnaire. Ces deux structures sont en effet celles modifiées au fur et à mesure du traitement de l'expression. Le dernier argument de \emph{foldl} est une fonction générale, \emph{process-token} qui gère l'action d'un token sur \emph{state}.  \\ Une fois le \emph{foldl} terminé, on renvoie le nombre restant sur la pile et le dictionnaire à \emph{repl}, qui affiche le résultat et se tient prêt à recevoir une nouvelle ligne.\\
Finalement, on a réussit à écrire notre calculatrice en Scheme de façon concise et modulaire, en faisant usage principalement d'appels à \emph{foldl} et de recherches avec continuations. 




\section{Résolution de problèmes de programmation}

\subsection{Analyse syntaxique d'une expression et\\traitement d'une expression de longueur quelconque}
	L'analyse syntaxique d'une expression postfixe est simple si on utilise une pile, comme c'est le cas ici. En effet, dans la notation postfixe, les opérateurs viennent toujours après leurs opérandes dans l'ordre de lecture de gauche à droite. Si on ajoute les nombres de l'expression à la pile en passant de gauche à droite, quand on tombe sur un opérateur, par exemple \emph{+}, on sait que les nombres à additionner sont les derniers à avoir été lus, et donc les nombres du dessus de la pile. Donc l'analyse consiste à retirer des nombres de la pile et à surveiller qu'on peut toujours en retirer le bon nombre. Si on analyse toute la ligne et qu'il ne reste qu'un nombre dans la pile au final, on saura que la syntaxe de la ligne aura été correcte, puisque à tout moment on aura trouvé le bon nombre d'opérandes pour nos opérateurs, et que de surcroit il n'y avait pas de chiffres superflus. \\ Pour effectuer ce processus, on utilise la fonction \emph{tokenize} pour transformer l'expression en une liste des termes qui y sont présents, sans les espaces. Chacun des \emph{tokens} est une liste qui contient les caractères du terme qu'il représente: par exemple, \emph{­123 2 +} devient \emph{((\SCMCAR1 \SCMCAR2 \SCMCAR3) (\SCMCAR2) (\SCMCAR+))}. La liste des tokens sera ensuite vidée token par token vers la pile. Cette pile est en fait une liste qu'on utilise comme une pile, en ne traitant que les valeurs sont au début de la liste. Ceci est aisé en Scheme car on peut considérer les listes comme des listes chaînées. Celles-ci facilitent aussi le traitement d'une expression de longueur quelconque, car la taille d'une liste chaînée est arbitraire. Pour vérifier que la pile contient assez d'opérandes, on utilisera \emph{length}. Pour retirer des éléments de la liste, on utilise la fonction \emph{split-at} qui sépare une liste en deux en retirant un certain nombre de données. 
\subsection{Calcul de l'expression}
	Pour calculer une expression, on applique les opérateurs trouvés sur le opérandes tirés de la pile dans la fonction \emph{process-token}. Pour savoir quelle fonction utiliser selon le token trouvé, on a défini une liste, appelée \emph{processors}, de couples de fonctions des tests et de manières d'opérer. Lorsqu'on cherche dans cette liste de couples, on applique toujours la première fonction sur le token. Si le test réussit, la recherche se conclut et on applique la deuxième fonction du couple sur \emph{state} et \emph{token}. Cette fonction génère le nouvel état.\\
\\
Les fonctions qui appliquent les opérations des tokens sont \emph{add-token-to-pile}, \emph{process-operator}, \emph{process-set} et \emph{process-ref}. Elles gèrent respectivement les nombres, les opérateurs (* + -), les assignations de variables et les références à des variables. Chaque opération recrée le nouvel état après l'application de celle-ci, ainsi:

\begin{description}

\item La fonction \emph{add-token-to-pile} ajoute le \emph{token}, identifié comme étant un nombre, à la pile.

\item La fonction \emph{process-operator} détermine les deux éléments du haut de la pile, la pile sans ces deux éléments et ajoute le résultat de l'opérateur correspondant au \emph{token} sur le dessus de la pile sans les opérandes. La fonction pour opérer est déterminée en faisant une recherche sur la liste associative des \emph{tokens} correspondants aux fonctions. On utilise les fonctions built-in d'addition, de soustraction et de multiplications, car elles gèrent déja les nombres de longueur arbitraire. 

\item La fonction \emph{process-set} ajoute le nom d'une variable et sa valeur au dictionnaire. 

\item La fonction \emph{process-ref} ajoute la valeur de la variable demandée à la pile.



\end{description}
\subsection{Affectation aux variables}
Pour l'affectation aux variables, comme pour le calcul, on détecte qu'une affectation est nécessaire grâce à la liste \emph{processors}. C'est donc \emph{process-set} qui effectue les étapes suivantes. La valeur de la nouvelle variable est toujours le nombre situé au haut de la pile, on commence donc par obtenir la valeur de la variable du dessus de la pile. \\ Les valeurs affectées aux variables sont contenues dans le dictionnaire de \emph{state}. Il faut faire attention de ne pas laisser dans le dictionnaire l'entrée correspondant à la vieille valeur de la variable à affecter. On a décidé de construire un nouveau dictionnaire qui ne contient pas de doublons. On part d'un nouveau dictionnaire ne contenant que la nouvelle entrée (le \emph{token} associé au nombre sur le dessus de la pile) et on opère un \emph{foldl} passant à travers l'ancien dictionnaire. On ajoute les entrées de celui-ci au nouveau seulement si leur clé est différente de notre token.\\
Pour accéder à la valeur d'une variable, on effectue une recherche avec continuation de la valeur associée à notre token dans le dictionnaire et on ajoute cette valeur à la pile. Ce processus est contenu dans \emph{process-ref}.

\subsection{Affichage des résultats et des erreurs}

Comme le \emph{repl} s'attend à recevoir une liste contenant une chaîne de caractères suivie du dictionnaire, on a simplement à reconstruire une telle structure une fois que le traitement de l'expression est complétée. Donc pour afficher les résultats, on dépile simplement le premier nombre de la pile et on l'attache devant le nouveau dictionnaire avec \emph{cons}. De la même façon, pour afficher le message d'erreur, on attache devant l'ancien dictionnaire le message à afficher.

\subsection{Traitement des erreurs}

Le traitement des erreurs est simple. On applique le principe du try...catch en utilisant les outils du langage. Notre méthode la plus englobante, \emph{traiter}, utilise la fonction \emph{with-exception-catcher}, qui prend deux fonctions; la fonction de \emph{catch} et de \emph{try}. Elle surveille l'exécution du \emph{try} et si une exception est lancée elle roule la fonction de \emph{catch}. Dans ce cas ci, chaque exception qu'on lance au cours de l'exécution donne un message d'erreur décrivant le problème. Dans le \emph{catch}, on ajoute ce message devant le dictionnaire pour qu'il soit imprimé à la place de la pile.\\
Nos définition de fonctions avec continuation qui peuvent échouer définissent deux continuations différentes; en cas de succès et en cas d'échecs. Dans le cas de \emph{find-cont} on définit l'échec comme une recherche infructueuse. Ces fonctions d'échec lancent le message d'erreur approprié au contexte avec \emph{raise}. Similairement, dans le cas de dépiler, par exemple quand vient temps d'effectuer une opération mathématique sur deux nombres ou d'attribuer une valeur à une variable, l'échec est définit comme un manque d'arguments.\\
On lance aussi une exception dans Traiter au moment de renvoyer la pile et le dictionnaire. On vérifie si la pile contient plus qu'un élément. Dans ce cas, on a une erreur de syntaxe car on a donné trop d'arguments à l'expression.

\section{Comparaison de notre expérience au TP1}

Il y a divers aspects de Scheme qui ont grandement simplifié la programmation vis-à-vis de C. Au final, notre programme Scheme fait environs 200 lignes et notre programme C, un peu plus de 600 lignes.\\
Une des parties les plus volumineuses de notre programme C est la gestion des nombres de taille arbitraire. Comme C ne propose que des nombres de taille fixe, environ 30\% de notre programme est consacré aux définitions des nombres et à leur gestion. En Scheme, rien de cela n'est nécessaire car on a accès à des nombres à précision infinie par défaut.\\
Une autre partie que Scheme gère automatiquement est la gestion de la mémoire; comme en C il faut allouer et désallouer les blocs de mémoires, une partie significative de notre code était dédiée à gérer cela et les erreurs découlant de ces opérations.\\
Finalement, un autre aspect où C est relativement déficient est la gestion des erreurs. Le langage ne propose pas de constructions pour lancer et attraper des erreurs, alors que la gestion de la mémoire impose de libérer correctement la mémoire lors des arrêts à mi-chemin des fonctions. On a donc que la plupart des fonctions sont complexifiées par les vérifications d'erreurs globales (l'alternative aurait été que les fonctions retournent d'une manière ou d'une autre les erreurs) et la libération partielle ou complète de bouts de mémoire.\\
\\
Les différences précédentes sont toutes attribuables au fait que Scheme est de plus haut niveau que C; une implémentation de la calculatrice en Python ou en Javascript aurait eu des bénéfices similaires. Une autre différence majeure est que notre code C est écrit de manière foncièrement impérative (il est relativement dur d'écrire du code C dans un style fonctionnel) alors que notre Scheme s'est limité à la partie fonctionnelle.\\
Ainsi, il est possible d'exprimer la totalité du programme comme une série de transformations pures sur une expression et un dictionnaire.\\
On peut donc décomposer le but du programme en parties plus petites jusqu'à ce qu'on puisse exprimer des fonctions simples et évidentes. Si une expression postfixe est une série d'opérations sur une pile, avec cette expression étant une suite de termes qui n'ont besoin que de la pile et du dictionnaire comme contexte pour générer le nouveau contexte, on peut donc raisonner comme suit:\\
À partir d'une liste de caractères représentant l'expression, si on veut opérer il faut d'abord transformer cette liste en liste de \emph{tokens}; il faut \emph{tokenizer}.\\
Pour pouvoir opérer des transformations sur les \emph{token}, il faut pouvoir identifier l'opération correspondante à chaque token.\\
Pour identifier et appliquer l'opération, il faut construire chacune des opérations de la calculatrice.\\
Pour chacune des opérations, il faut un test qui dit simplement si on peut opérer ou non et il faut une fonction qui effectue la transformation du token sur le contexte.\\
Et ainsi de suite: Éventuellement on arrive à des fonctions simples comme  \emph{number?} et \emph{add-to-pile}.\\
\\
Les opérations définies sur des ensembles du programme sont quasiment toutes définies sous forme récursive itérative: principalement, si une opération se porte bien à être définie à l'aide d'un \emph{foldl}, c'est ce qui est utilisé. Par exemple, et comme mentionné précédemment, le traitement de l'expression est exprimé comme un \emph{foldl} sur les \emph{tokens} avec chaque étape étant l'application de la bonne opération. Sinon, un autre exemple de récursivité est la recherche; que ce soit sur le dictionnaire pour trouver les variables ou sur les types d'opérations pour un \emph{token} quelconque.\\
\\
En ce qui concerne les continuations, elles sont utilisées principalement afin de marquer des types de résultats différents; pour \emph{find} on définit un argument de continuation en cas de succès (\emph{cont}) et un argument de continuation en cas d'échec (\emph{fail}). Il est alors assez simple de définir diverses manières de gérer des résultats sans avoir à vérifier la nature de l'argument. Dans le cas de \emph{split-at}, qui est utilisé pour définir \emph{depile}, la continuation nous permet de définir un retour de plusieurs arguments de la fonction: Soit les éléments de tête, pour lesquelles que l'argument accepte et les éléments de queue, qui sont ceux à partir du premier élément refusé. Sur la fonction \emph{depile} on a alors les éléments dépilés et le reste de la pile.\\
\\
Finalement, comme la gestion des erreurs se fait via les exceptions et que notre cas typique est de lever une exception lorsqu'une opération avec continuation échoue, nous avons définit une macro permettant de construire simplement la gestion du cas. Il faut à chaque fois construire une fonction qui lance l'erreur avec un message construit dynamiquement basé sur le token. Une macro simple qui prend un ensemble d'éléments à concaténer pour avoir le message final est assez avantageuse par rapport à l'emploi d'une fonction simple: on ne veut évaluer le message que si l'opération a échoué et une macro nous évite d'avoir à écrire un nouveau lambda à chaque fois.

\end{document}
